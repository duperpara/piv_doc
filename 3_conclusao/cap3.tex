\chapter{Conclusão}

As considerações finais formam a parte final (fechamento) do texto, sendo dito de forma resumida (1) o que foi desenvolvido no presente trabalho e quais os resultados do mesmo, (2) o que se pôde concluir após o desenvolvimento bem como as principais contribuições do trabalho, e (3) perspectivas para o desenvolvimento de trabalhos futuros. O texto referente às considerações finais do autor deve salientar a extensão e os resultados da contribuição do trabalho e os argumentos utilizados estar baseados em dados comprovados e fundamentados nos resultados e na discussão do texto, contendo deduções lógicas correspondentes aos objetivos do trabalho, propostos inicialmente.

Antes de ir para as Referências, devo dizer que para obtê-las, você deve abrir o arquivo referencias.bib, neste arquivo estão as referências no formato \textit{BibTex}. Para que as citações apareçam no seu trabalho você deve:
\begin{itemize}
	\item pesquisar sobre o livro que você quer citar no site \textit{scholar.google.com.br}.
	\item colocar o nome do título do livro o qual você procura.
\end{itemize}
