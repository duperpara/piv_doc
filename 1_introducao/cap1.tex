\chapter{Introdução}
% para fazer marcações no texto para eventuais referências, utilize o o label
% variando entre SEC FIG IMAG TAB ALG
\label{sec:introdução}

%para inicializar o primeiro parágrafo de uma seção utilize o padrão a seguir
% onde a letra na chave {A} é a letra em questão que aparecerá 
\lettrine[lines=3]{O}{} grupo pretende neste projeto construir uma base que possa manter refrigerado um recipiente contendo alguma bebida. Desse modo, foi pensado em utilizar uma placa peltier, que em conjunto com um dissipador de calor poderá prover a habilidade térmica procurada.

\section{Contextualização do tema}
	
\hspace{11mm} Este projeto trabalha no controle e automação de uma planta de nível de água didática, constituída de dois tanques e controlada por um Controlador Lógico Programável (CLP).  O sistema tem como objetivo demonstrar as propriedades de um processo de simples entrada e simples saída (SISO - simple input, simple output).

%%SILVA, Rodrigo Allan, DIDACTIC MODULE DEVELOPMENT OF LEVEL CONTROL. 72 f. Undergraduate Final Project – Undergraduate Degree in Electrical Engineering. Federal Technological University of Paraná. Cornélio Procópio, 2014.

\hspace{11mm} Sistemas de controle de nível são de grande importância dentro do cenário industrial. De acordo com Rodrigo Silva (2014), processos que envolvam o uso ou até a produção de petróleo, utilizam um controle de nível e dependem de sua eficiência para garantir a segurança tanto do equipamento, como dos técnicos envolvidos no processo.

\hspace{11mm}Embora o conceito deste projeto seja simples, o processo de controle de nível pode ter uma maior complexidade, principalmente quando o processo demanda uma vazão de entrada ou saída variáveis. Assim sendo, é importante que sejam feitos estudos para a implementação e dimensionamento de um controle eficiente, especialmente em um ambiente educacional com um enfoque em aproximar o aluno de um ambiente profissional, ao passo que aplica seus conhecimentos.


%%Corrigido
\section{Objetivos}

\hspace{11mm}Dentre os objetivos do projeto integrador será realizado o modelamento de uma planta de nível, a elaboração de um controlador para o sistema. Ao final do projeto, será feito um sistema supervisório, que será utilizado para monitoramento e controle da variável de referência.

\hspace{11mm}Além de controlar a planta este projeto tem o intuito de que os envolvidos possam aplicar e aprofundar seus conhecimentos adquiridos nas disciplinas de Controle de Processos, Informática Industrial e Técnicas de Automação Industrial. Para tal finalidade, foram definidos os seguintes objetivos específicos:

    \begin{itemize}
        \item Estudar o processo;
        
        \item Realizar Modelagem matemática do sistema;
        
        \item Executar manutenções ou ajustes necessários na planta;
        
        \item Efetuar testes em malha aberta para identificação de constantes;
    
        \item Identificar um modelo de controle.
        
    \end{itemize}