
% ---
% Pacotes básicos
% ---
\usepackage{bookmark}				% Usa a fonte Bookman Old Style
\usepackage[T1]{fontenc}			% Selecao de codigos de fonte.
\usepackage[utf8]{inputenc}		% Codificacao do documento (conversão automática dos acentos)
\usepackage{color}				% Controle das cores
\usepackage{graphicx}			% Inclusão de gráficos
\usepackage{microtype} 			% para melhorias de justificação
\usepackage{amsmath}
\usepackage[brazilian,hyperpageref]{backref}	 % Paginas com as citações na bibl
\usepackage[alf]{abntex2cite}	% Citações padrão ABNT
\usepackage{float}
\usepackage{url}
\usepackage{enumerate}
\usepackage{siunitx}
\usepackage{setspace}
\usepackage[euler]{textgreek}

%%%%%%%%% NOVO
\usepackage{hyperref}
\hypersetup{
    colorlinks=true,
    linkcolor=blue,
    citecolor=blue,
    filecolor=magenta,      
    urlcolor=blue,
    bookmarksopen=true,
}
\urlstyle{same}

%%%   letra capitular
\usepackage{lettrine}
\usepackage{palatino}

%%%    tabelas - configuração
\usepackage{booktabs}
\usepackage{siunitx}

\usepackage{lscape}


%%% lista de abreviações e siglas automática 
\usepackage{acro}


%%%%%% estilo do capítulo
%%%%%% escolha 1 e tire o comentário
%% acesse a página e escolha
%%http://ctan.math.washington.edu/tex-archive/info/latex-samples/MemoirChapStyles/MemoirChapStyles.pdf

%% primeiro
%\chapterstyle{madsen}

%% segundo
%\chapterstyle{southall}

%% terceiro
%\chapterstyle{Ger}

%% quarto
%\chapterstyle{verville}

%% quinto
\setlength\midchapskip{10pt}
\makechapterstyle{VZ23}{
    \renewcommand\chapternamenum{}
    \renewcommand\printchaptername{}
    \renewcommand\chapnumfont{\Huge\bfseries\centering}
    \renewcommand\chaptitlefont{\Huge\scshape\centering}
    \renewcommand\afterchapternum{%
        \par\nobreak\vskip\midchapskip\hrule\vskip\midchapskip}
    \renewcommand\printchapternonum{%
        \vphantom{\chapnumfont \thechapter}
        \par\nobreak\vskip\midchapskip\hrule\vskip\midchapskip}
}
\chapterstyle{VZ23}

%%%%%%%%%%%%%

\titulo{Projeto Integrador V}
\autor{Henrique G. Silvério \\ Luiz F. Niquelatte \\ Matheus G. Zweibrucker}
\local{Florianópolis - Santa Catarina}
\data{Setembro de 2022}
\instituicao{%
  Instituto Federal de Santa Catarina - IFSC
  \par
  Campus Florianópolis}
\preambulo{Trabalho submetido à avaliação, como requisito parcial, para a obtenção de nota na disciplina de Projeto Integrador, ministrada pelos professores Francisco Rafael Moreira da Mota e Valdir Noll}


% Informações do PDF
\makeatletter
\hypersetup{
    	%pagebackref=true,
	pdftitle={\@title}, 
	pdfauthor={\@author},
    	pdfsubject={\imprimirpreambulo},
    pdfcreator={LaTeX with abnTeX2},
	pdfkeywords={abnt}{latex}{abntex}{abntex2}{trabalho acadêmico},
	bookmarksdepth=4
}
\makeatother

% --- 
% Espaçamentos entre linhas e parágrafos 
% --- 

% O tamanho do parágrafo é dado por:
\setlength{\parindent}{1.3cm}

% Controle do espaçamento entre um parágrafo e outro:
%\setlength{\parskip}{0.2cm}  % tente também \onelineskip

%\setbeforesecskip{3em}
%\setbeforesubsecskip{3em}

% ---
% compila o indice
% ---
\makeindex